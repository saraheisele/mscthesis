\pdfbookmark[0]{Introduction}{pdf.intro}

\noindent
These charts compare the STIX 2.10 against Unicode 13.0.0.

\ifSTIXMath

\paragraph{Note}

The STIX Two Math font includes duplicates of all characters supported
in STIX Two Text Regular, but in some cases with different forms
suited to use in math typesetting. This character coverage ensures
that any character from this common subset that occurs within a math
equation will be correctly displayed, even if it is not a math symbol
with a generally recognised meaning.  However, use of the STIX Two
Math font to typeset non-mathematical text is discouraged, and the
Text font family should be used for this purpose.

\fi

\subsubsection*{Legend}

\begin{itemize}

\item Characters shown in black (e.g., U+0021) are supported directly by
the font.

\item Characters shown in \textcolor{red}{red} (e.g., U+01CE) are not
  directly supported but can be synthesized by a Unicode-aware shaping
  engine.  (These charts were generated
  with \XeTeX\ \number\XeTeXversion\XeTeXrevision.)

% \textbf{Note}: Many of the characters in this category are poorly
%   rendered because of deficiencies in the font's \textsc{gpos} table.
%   Plans are being made to address these problems.

\ifshowghosts
\item Characters shown in \textcolor{gray}{gray} are not supported by
  STIX, but are borrowed from Cambria for demonstration purposes.
\fi

\item Grey squares (e.g., U+0380) indicate code points that are not
    assigned a meaning in Unicode 13.0.

\item Otherwise blank squares that contain a code point (e.g., U+0020)
indicate characters with no visible representation.  These may or may
not be zero-width characters; see the Unicode Standard for details.

\item Completely blank squares (e.g., U+0181) indicate code points
  correspondinng to characters that STIX 2.10 does not support.

\item Black squares (e.g., U+FFFE) indicate noncharacter code points.

\end{itemize}

\subsubsection*{Notes}

\begin{itemize}

\item Not every composite character supported by the STIX fonts is
  included in these charts; only characters that are assigned a code
  point by Unicode~13.0 are shown.  For example, you will not find
  ``x̣'' in these charts since Unicode does not assign a code point to
  that character.  It can, however, be composed from the sequence U+0078
  LATIN SMALL LETTER X followed by U+0323 COMBINING DOT BELOW.

% \item Similarly, not every variant glyph of every character is displayed.
%   See the STIX 2.10 release notes and accompanying documentation for
%   more information on the OpenType font features and stylistic sets
%   supported by STIX Two.

\item Glyphs are shown at size 20\,pt unless otherwise indicated.

\end{itemize}

\endinput
